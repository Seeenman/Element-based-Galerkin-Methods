\documentclass[10pt]{article}
\usepackage{float}
\usepackage{amsmath}
\usepackage{graphicx}
\input{epsf}
%\usepackage{a4}

\newtheorem{theorem}{Theorem}
\newtheorem{acknowledgement}[theorem]{Acknowledgement}
\newtheorem{algorithm}[theorem]{Algorithm}
\newtheorem{axiom}[theorem]{Axiom}
\newtheorem{case}[theorem]{Case}
\newtheorem{claim}[theorem]{Claim}
\newtheorem{conclusion}[theorem]{Conclusion}
\newtheorem{condition}[theorem]{Condition}
\newtheorem{conjecture}[theorem]{Conjecture}
\newtheorem{corollary}[theorem]{Corollary}
\newtheorem{criterion}[theorem]{Criterion}
\newtheorem{definition}[theorem]{Definition}
\newtheorem{example}[theorem]{Example}
\newtheorem{exercise}[theorem]{Exercise}
\newtheorem{lemma}[theorem]{Lemma}
\newtheorem{notation}[theorem]{Notation}
\newtheorem{problem}[theorem]{Problem}
\newtheorem{proposition}[theorem]{Proposition}
\newtheorem{remark}[theorem]{Remark}
\newtheorem{solution}[theorem]{Solution}
\newtheorem{summary}[theorem]{Summary}
\newenvironment{proof}[1][Proof]{\textbf{#1.} }{\ \rule{0.5em}{0.5em}}
%\input{tcilatex}

%FXG Commands
\newtheorem{guess}{Definition}
%\newcommand{\diff2}[2] {\frac{\partial^2 #1}{ \partial {#2}^2}}
%\newcommand{\diff2}[2] {\frac{\partial #1}{\partial #2}}
\newcommand{\Norder} {N}
\newcommand{\order}{\mathcal{O}}
\newcommand{\Npoints} {N_p}
\newcommand{\diff}[2] {\frac{\partial #1}{\partial #2}}
\newcommand{\dxx}[2] {\frac{\partial^2 #1}{\partial {#2}^2}}
\newcommand{\difft}[2] {\frac{d #1}{d #2}}
\newcommand{\lagrange}[1] {\frac{d #1}{dt}}
\newcommand{\lebesgue}{\parallel I \parallel}
\newcommand{\polysp}{\mathcal{P}_N}
\newcommand{\vc}[1]{\mbox{\boldmath$#1$\unboldmath}}
\newcommand{\grad}{\vc{\nabla}}
\newcommand{\inte}{\int_{\mbox{\footnotesize ${\Omega_e}$}}}
\newcommand{\intce}{\int_{\mbox{\footnotesize ${\widehat{\Omega}_e}$}}}
\newcommand{\intb}{\int_{\mbox{\footnotesize ${\Gamma_e}$}}}
\newcommand{\intcb}{\int_{\mbox{\footnotesize ${\widehat{\Gamma}_e}$}}}
\newcommand{\inth}{\int_{\mbox{\footnotesize ${\Omega}$}}}
\newcommand{\inthb}{\int_{\mbox{\footnotesize ${\Gamma}$}}}
\newcommand{\intv}{\int_{\mbox{\footnotesize ${\sigma}$}}}
\newcommand{\sumv}{\sum_{K=1}^{N_{\mathrm{lev}}}}
\newcommand{\sumk}{\sum_{L=1}^{K}}
\newcommand{\half}{\frac{1}{2}}
\newcommand{\inti}{\int_{\mbox{\footnotesize\sf I}}}
\newcommand{\intbd}{\oint_{\mbox{\footnotesize ${\delta}$\sf D}}}
\newcommand{\intbi}{\oint_{\mbox{\footnotesize ${\delta}$\sf I}}}
\newcommand{\ldnorm}[1]{\left\| #1 \right\|_{\mbox{\footnotesize \sf D}} }
\newcommand{\lonorm}[1]{\left\| #1 \right\|_{\Omega}}
\newcommand{\spc}[1]{\mbox{\sf #1}}
\newcommand{\ope}[1]{{\cal #1}}
\newcommand{\mt}[1]{{\rm #1}}
\newcommand{\dis}{\displaystyle}
\newcommand{\ve}{\varepsilon}
\newcommand{\ov}{\overline}
\newcommand{\wt}{\widetilde}
\newcommand{\wh}{\widehat}
\newcommand{\be}{\begin{equation}}
\newcommand{\ee}{\end{equation}}
\def\bepsilon{\mbox{\boldmath $\epsilon $}}
\def\bpsi{\mbox{\boldmath $\psi $}}
\def\bphi{\mbox{\boldmath $\phi $}}
\def\bmu{\mbox{\boldmath $\mu $}}
\def\Et{ \tilde{E} }
\def\Ht{ \tilde{H} }
\def\sdot{ \dot{\sigma} }
\newcommand{\innerd}[2]{\left( #1,#2 \right)_{\mbox{\footnotesize \sf D}}}
\newcommand{\inners}[2]{\left( #1,#2 \right)_{\mbox{\footnotesize
${\delta}$\sf D}}}
\newcommand{\innerbd}[2]{\left( #1,#2 \right)_{\mbox{\footnotesize ${\delta}$\sf
 D}}}
\newcommand{\innerO}[2]{\left( #1,#2 \right)_{\Omega}}
\newcommand{\innerOs}[2]{\left( #1,#2 \right)_{\delta \Omega}}
\newcommand{\innerdk}[2]{\left( #1,#2 \right)_{\mbox{\footnotesize \sf D}^k}}
\newcommand{\intbdk}{\oint_{\mbox{\footnotesize ${\delta}$\sf D}^k}}
\newcommand{\ldnormk}[1]{\left\| #1 \right\|_{\mbox{\footnotesize \sf D}^k}}
\newcommand{\intdk}{\int_{\mbox{\footnotesize \sf D}^k}}
\newcommand{\epsD}{\varepsilon_{\mbox{\footnotesize \sf D}}}
\newcommand{\ldnormsob}[2]{\left\| #2 \right\|_{W^{#1}(\mbox{\footnotesize \sf D
})}}
\newcommand{\lbdnorm}[1]{\left\| #1 \right\|_{\mbox{\footnotesize \sf $\delta$D}
}}
\renewcommand{\thetable}{\Roman{table}}
\newcommand{\qvector}{\vc{q}}

\DeclareMathOperator{\Span}{span}
\DeclareMathOperator{\Dim}{dim}

\newcommand{\polyquad}{\mathcal{Q}_{N}}
\newcommand{\polyP}{\mathcal{P}_{N}}
\newcommand{\polyPnpm}{\mathcal{P}_{(N+M)}}
\newcommand{\polyPd}{\mathcal{P}_{d}}
\newcommand{\polyPnm}{\mathcal{P}_{N,M}}
\newcommand{\polyPn}{\mathcal{P}_{N,0}}
\newcommand{\transpose}{^{\mathcal{T}}}

\begin{document}
\title{MA4245 Mathematical Principles of Galerkin Methods \\
Project 3: 2D Poisson Equation}
\author{Prof. Frank Giraldo \\
Department of Applied Mathematics \\
Naval Postgraduate School \\
Monterey, CA 93943-5216}
\date{Due: May, 20 2021 at 1pm}
\maketitle

\section{Continuous Problem}
The governing partial differential equation (PDE) is
\[
\grad^2 q(x,y) = f(x,y) \qquad \forall (x,y) \in [-1,1]^2
\]
where $f=-2 \pi^2 \sin \pi x \sin \pi y$ which yields the analytic solution $q(x,y) = \sin \pi x \sin \pi y$. 
Clearly, this problem represents a Poisson problem which is a generalized elliptic equation and is, therefore, a boundary value problem. 

\section{Simulations}
Use the CG  method to solve this equation. 
You need to write the code to handle arbitrarily-sized elements, polynomial orders, and integration formulas. What I mean by arbitrarily-sized elements is that you should not assume that each element is of the same size. 
You can, however, assume that each element uses the same polynomial and integration orders.

\subsection{Results You Need to Show}
You must show results for linear elements $N=1$ with increasing number of elements $N_e$ and then show results for 
$N=2$, $N=4$ , $N=8$, and $N=16$ with increasing numbers of elements. Plot $L^2$ error norms (defined below) versus number of points ($N_P$) and show all 5 curves on one plot. Remember that you must use 
a log plot for the error to capture the spectral convergence. To confirm that you have not assumed anything specific about the geometry, use the rotate\_grid (matlab) or warp\_grid (julia) in the driver files to rotate the grid.  You should expect similar results to the unrotated grid but will not be identical.  Show me a plot with the error for one grid configuration (e.g., $N=4, \,Q=5, \, N_e=4$).

For the following simulations, you must turn in two plots: one for exact integration and another for inexact. Write a discussion 
on your findings.

\paragraph{N=1 Simulations}
For $N=1$ use $N_e=4, 8, 16, 24$, and 32 elements.

\paragraph{N=2 Simulations}
For $N=2$ use $N_e=2, 4, 8, 12$ and 16 elements.

\paragraph{N=4 Simulations}
For $N=4$ use $N_e=1, 2, 4, 6$, and 8 elements.

\paragraph{N=8 Simulations}
For $N=8$ use $N_e=1, 2, 3$, and 4 elements.

\paragraph{N=16 Simulations}
For $N=16$ use $N_e=1$ and 2 elements.

Here is what the analytic and numerical solutions should look like:
\begin{figure}[h]
\begin{center}
\begin{minipage}{2.20in}
\includegraphics[width=2.20in]{../../Figures/2d_elliptic_cg/CG_2d_elliptic_analytic_n04p16.eps}
a) Analytic Solution
\end{minipage} \ \hspace{0.125in} \
\begin{minipage}{2.20in}
\includegraphics[width=2.20in]{../../Figures/2d_elliptic_cg/CG_2d_elliptic_num_n04p16.eps}
b) CG Numerical Solution
\end{minipage} 
\caption{The 2D Elliptic equation solutions for a) analytic and b) CG numerical solutions using $n_r=4$ elements in each direction ($N_e=16$) and $N=16$ order 
polynomials.}
\end{center}
\end{figure}

Here is an example of the kind of plot you should show me:
\begin{figure}[h]
\begin{center}
\begin{minipage}{2.20in}
\includegraphics[width=2.20in]{../../Figures/2d_elliptic_cg/CG_2d_elliptic_exact_ConvergenceRates.eps}
a) Exact Integration 
\end{minipage} \ \hspace{0.125in} \
\begin{minipage}{2.20in}
\includegraphics[width=2.20in]{../../Figures/2d_elliptic_cg/CG_2d_elliptic_inexact_ConvergenceRates.eps}
b) Inexact Integration
\end{minipage} 
\caption{The Convergence Rates for CG using a) exact and b) inexact integration.}
\end{center}
\end{figure}

\clearpage

\section{The Write-Up}
In your write-up, please include the convergence rate plots as above and give a detailed discussion of how you solved the 
problem. E.g, how did you construct the basis functions, mass matrix, and Laplacian matrix, and how you imposed the 
homogeneous Dirichlet boundary conditions. Tell me what you learned about using CG for elliptic problems. Also include a complexity analysis 
of your code (i.e., how many mathematical floating point operations the code required between exact and inexact integration). Assuming 
that you could write the code only for inexact integration, how much cheaper would that be?

\section{Helpful Relations}

\paragraph{Error Norm}
The normalized L2 error norm that you should use is:
\be
||error||_{L^2} = \sqrt{ \frac{ \sum_{k=1}^{N_P}  \left( q^{numerical}_k - q^{exact}_k \right)^2}{ \sum_{k=1}^{N_P} \left( q^{exact}_k \right) ^2} }
\ee
where $k=1,...,N_P$ are $N_p=(N_e N + 1)^2 $ global gridpoints and $q^{numerical}$ and $q^{exact}$ are the numerical and exact solutions.

\end{document}
